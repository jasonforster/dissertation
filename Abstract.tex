Crystalline packings of dielectric spheres with diameters of 100 to 400 nm can produce materials with photonic band gaps and are called photonic crystals.
The periodic structure of these materials gives rise to constructive interference of visible light and these crystals can be brilliantly colored.
Similarly, periodic structures in bird feathers and insect scales are responsible for many of the most striking colors found in the natural world.
The basic scattering mechanism behind these iridescent colors is relatively straight forward and well described by Bragg scattering.
Not nearly as well understood are the non-iridescent colors found in many bird feathers that are the result of scattering from disordered nano-structures.
In this dissertation I aim to expand our understanding of structural color in two ways.
First, I describe the design, assembly and characterization of isotropic structurally colored films of colloidal spheres and compare them to the non-iridescent structural colors of bird feathers.
Second, I describe a new type of photonic crystal composed of dumbbell-shaped colloidal particles that combines the features of field switchable structural color and birefringence. 