The interaction of light and matter has held the interest of humans for millennia.
A desire to understand this relationship has motivated physicists to study the fundamental nature of light and matter, enabled engineers to design and build myriad useful materials and devices, and given artists the tools to create so many culturally important and aesthetically pleasing works of art.
Relevant to this dissertation are the methods for producing a material that reflects, scatters or absorbs light in the visible portion of the electromagnetic radiation spectrum.
Specifically, I aim to broaden our understanding of light interacting with materials possessing a fine physical structure with characteristic dimensions on the order of hundreds of nanometers.

The property of a material that determines its interaction with light is the electric permittivity function, $\epsilon$, which in general depends on the frequency of light, $\epsilon(\omega)$.
Scattering, reflection, and refraction take place whenever light encounters an interface defined by a sudden difference in $\epsilon(\omega)$.
Absorption can occur if $\epsilon(\omega)$ has an imaginary component.
The index of refraction, $n$, of a material is a useful quantity to describe its interaction with light and is related to $\epsilon$ in a simple way, $n(\omega) = \sqrt{\epsilon(\omega)}$.
Most of the materials I will describe in this dissertation are dielectric and thus do not have significant absorption at optical frequencies.
Furthermore, I will largely ignore the frequency dependence of the index  of refraction because for the materials used here the value only varies by a few percent across the visible spectrum.

The particular form of an interface between differing indices of refraction has a dramatic effect on its scattering properties.
When the interface is smooth compared to the wavelength of light we use the words reflection and refraction to describe the scattering.
Smooth interfaces are responsible for phenomena like the reflected image of a mountain from the water/air interface of an alpine lake and the focusing of a collimated beam of light to a spot by a plano-convex lens.

More complicated scattering takes place when a material contains multiple interfaces.
If $n$ varies in a regular way and with a period comparable to the wavelength of light, scattering can lead to constructive interference of certain wavelengths of light and give the material a color --- a structural color.
This type of variation can be periodic in one, two, or three dimensions.
Examples of this type of structural color are found in insect scales and bird feathers \cite{Srinivasarao:1999p9327, Vukusic:2003p1380, Parker:2007p11434, Welch:2007p1313, Kinoshita:2008p10935}, as well as in opal gemstones \cite{Sanders1968}.
Man-made materials with this property are often called photonic crystals.
The scattering from these structures produces colors that change when viewed or illuminated from different directions, a phenomenon known as iridescence.

If the index of refraction varies in a disorganized way with no translational order the scattering of light is less discriminate.
When the spatial distribution of scatterers is completely disordered, the material will appear white because it scatters all wavelengths of light with roughly equal efficiency \cite{Born:1999}.
Dilute colloidal suspensions such as milk appear white for this reason.
However, systems with no translational order can have a structural color if the individual scatterers have a strong wavelength dependence.
For example, if the scatterers are much smaller than the wavelength of light, the intensity of scattered light is inversely proportional to the fourth power of the wavelength, $I\propto \lambda^{-4}$ \cite{Born:1999}.
This so-called Rayleigh scattering takes place as sunlight interacts with fluctuations in the local density of the gasses in Earth's atmosphere and explains why the sky away from the sun appears blue and the sun itself appears yellow during the day and red at sunset.

There are some materials that appear structurally disordered and yet selectively scatter a narrow range of wavelengths.
The feathers of many birds contain one of two types of disordered scattering structures \cite{Dufresne:2009p6342}. 
Both types are composed of air and a dried protein matrix of $\beta$-keratin.
One of these structures is a bi-continuous network of air channels and $\beta$-keratin channels.
The other is an array of spherical air inclusions in a background of $\beta$-keratin.
At first glance, these structures appear to be completely disordered, but Fourier Transform analysis of electron-microscopy images \cite{Prum:1998p1228} or small-angle X-ray scattering (SAXS) experiments \cite{Saranathan:2011} reveal that they possess a characteristic length scale without long range translational order.
A consequence of this structure is that the colors produced are not iridescent --- they appear to be the same from all angles.
The development of these structures and the scattering mechanisms by which they produce color has been the subject of recent study \cite{Prum:1998p1228, Dufresne:2009p6342, Prum:2009p3119, Noh:2009AM, Saranathan:2011}.

A major focus of this dissertation is the design, assembly, and characterization of films that mimic these characteristics.
The material we will use to assemble these structures are polymer colloidal spheres.
This choice is based a number of factors, including the following: methods for synthesizing monodisperse spherical polymer colloids are well-established and have been applied in academia and industry for decades, it is possible to precisely control the size of the particles across a broad range of length-scales, and their rapid assembly into densely packed structures is aided by thermal fluctuations.

While making spherical colloids in large quantities has been possible for years, only recently have techniques for synthesizing sub-micron colloids with anisotropic shapes and interactions been developed \cite{Perro2005, Jiang2008, Park2010, Duguet2011, Kuijk2011}.
The second major focus of this dissertation takes advantage of this new-found capability by using dumbbell-shaped particles to assemble a photonic crystal with optical properties that cannot be replicated by a crystal of spheres.

In Chapter~\ref{chap:sphere-isotropic}, I discuss the concepts relevant to photonic crystals and contrast them with materials possessing an isotropic structural color.
First, I describe naturally occurring isotropic structural colors found in bird feathers in detail.
Then I then describe the design, assembly, and characterization of materials composed of spherical colloidal particles that produce an isotropic structural color.

In Chapter~\ref{chap:dumbbell-crystal}, I describe a new type of photonic crystal composed of dumbbell-shaped colloidal particles.
The anisotropic shape of the dumbbells allows us to make a field switchable birefringent photonic crystal.

In Chapter~\ref{chap:conclusions}, I summarize the results and implications of the work in this dissertation and discuss possible directions for future research.
First, I propose methods for improving the optical performance and functionality of isotropic structural color materials by inverting the structure formed by the spheres and applying an electric field.
Then I propose new experiments to determine the crystal structure of dumbbells in an electric field and the potential for using dumbbells as colloidal surfactants.


